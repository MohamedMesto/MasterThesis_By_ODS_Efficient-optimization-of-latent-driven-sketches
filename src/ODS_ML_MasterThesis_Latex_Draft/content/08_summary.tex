\chapter{Summary}
\label{chapter:summary}
In this work, our goal is to implement an ML-hybrid system that excels at recognizing high-frequency internet traffic. At the same time, our system should be able to be aware of and deal with concept drift.

To effectively extract information from input features, we choose RNN, to be more specifically LSTM, as our principal layer for  extracting feature bit by bit. After that, we add dense layers on top to predict counts. Besides counts prediction, we also pay much attention on classification -- if our model can classify between top items and bottom items. Therefore, we utilize AUC as our primary metric for performance evaluation.

To capture and address concept drift, we preprocess data collected from CAIDA in a chronological order so that our data is in forms of consecutive time windows. We succeed in spotting concept drift by training on older windows and testing on newer windows. Afterwards, we manage to reduce concept drift by making use of adaptive learning. We let the model train on recent data so that the model is aware of latest patterns and getting dynamic enough. Eventually, the model performs much better on the test data.

In addition to ML model, we integrate sketch algorithms (for example, count-min) into our system. Instead of feeding the ground-truth labels to the NN, we feed the counts (the output of sketches) that we get from the whole
system. Through this method, we perform semi-supervised learning on our model and reach a new high of AUC score.

Apart from temporal patterns, we try to represent items from more latent to more explicit and thus prioritize Heavy-Hitters. We design a top k pyramid pooling scheme as well as alternate it with dense layers. As a result, our pyramid model is able to predict item counts with a decent accuracy. 

We believe that our ML-hybrid system is well-rounded, that is, excel at not only adapting to dynamic temporal environment, but also identifying heavy traffic and predicting counts. Due to the flexibility of our architecture, we believe it will be convenient to include more advanced techniques into our model to enhance its performance, allowing us to develop our model beyond its current restrictions and fulfill its visions in the future. 